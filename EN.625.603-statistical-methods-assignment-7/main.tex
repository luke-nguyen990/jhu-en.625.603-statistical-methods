\documentclass{uofa-eng-assignment}
\usepackage{amsmath}
\usepackage{enumerate}% http://ctan.org/pkg/enumerate
\usepackage{lipsum}
\usepackage{hyperref}
\usepackage{amsmath, amsthm, amssymb, amsfonts, physics}
\usepackage{mathtools}
\usepackage{graphicx}
\usepackage{fdsymbol}

\hypersetup{
    colorlinks=true,
    linkcolor=blue,
    filecolor=magenta,
    urlcolor=cyan,
    pdftitle={Overleaf Example},
    pdfpagemode=FullScreen,
}

\graphicspath{ {./images/} }

\DeclareRobustCommand{\rchi}{{\mathpalette\irchi\relax}}
\newcommand{\infdiv}{D\infdivx}
\newcommand{\irchi}[2]{\raisebox{\depth}{$#1\chi$}} % inner command, used by \rchi
\newcommand\aug{\fboxsep=-\fboxrule\!\!\!\fbox{\strut}\!\!\!}
\newcommand*{\name}{\textbf{Luke Nguyen}}
\newcommand*{\id}{\textbf{D5850A}}
\newcommand*{\course}{Statistical Methods and Data Analysis (EN.625.603)}
\newcommand*{\assignment}{Problem Set 8}

\begin{document} \maketitle
%%%%%%%%%%%%%%%%%%%%%%%%%%%%%%%%%%%%%%%%%%%%%%%%%%%%%%%%%%%%%%%%%%%%%%%%%%%%%%%%%%%%%%%%%%%%%%%%%%%%    
\begin{enumerate}
    %%%%%%%%%%%%%%%%%%%%%%%%%%%%%%%%%%%%%%%%%%%%%%%%%%%%%%%%%%%%%%%%%%%%%%%%%%%%%%%%%%%%%%%%%%%%%%%%%%%%    
    %%%%%%%%%%%%%%%%%%%%%%%%%%%%%%%%%%%%%%%%%%%%%%%%%%%%%%%%%%%%%%%%%%%%%%%%%%%%%%%%%%%%%%%%%%%%%%%%%%%%    
    %%%%%%%%%%%%%%%%%%%%%%%%%%%%%%%%%%%%%%%%%%%%%%%%%%%%%%%%%%%%%%%%%%%%%%%%%%%%%%%%%%%%%%%%%%%%%%%%%%%%    
    %%%%%%%%%%%%%%%%%%%%%%%%%%%%%%%%%%%%%%%%%%%%%%%%%%%%%%%%%%%%%%%%%%%%%%%%%%%%%%%%%%%%%%%%%%%%%%%%%%%%    
    %%%%%%%%%%%%%%%%%%%%%%%%%%%%%%%%%%%%%%%%%%%%%%%%%%%%%%%%%%%%%%%%%%%%%%%%%%%%%%%%%%%%%%%%%%%%%%%%%%%%    
    \item[]
        \textbf{Question 9.2.12} \\
        Suppose that $H_0: \mu_X = \mu_Y$ is being tested against $H_1:\mu_X \neq \mu_Y$,
        where $\    sigma_{x}^{2}$ and $\sigma_{Y}^{2}$ are known to be 17.6 and 22.9, respectively.
        If $n=10$, $m=20$, $\bar{x} = 81.6$, and $\bar{y} = 79.9$,
        what $P-$value would be associated with the observed $Z$ ratio? \\
        \textbf{Solution} \\
        Applying \textbf{Theorem 9.2.3}, test statistic as follows:
        \begin{align*}
            Z              & = \frac{(\bar{x} - \bar{y})}{\sqrt{\frac{\sigma^2_X}{n} + \frac{\sigma^2_Y}{m}}} \\
                           & = \frac{81.6 - 79.7}{\sqrt{\frac{17.6}{10} + \frac{29.9}{20}}}                   \\
                           & \approx 0.9974                                                                   \\
            \text{P-value} & = 2\times 0.1611                                                                 \\
                           & = 0.3222
        \end{align*}
        %%%%%%%%%%%%%%%%%%%%%%%%%%%%%%%%%%%%%%%%%%%%%%%%%%%%%%%%%%%%%%%%%%%%%%%%%%%%%%%%%%%%%%%%%%%%%%%%%%%%    
        %%%%%%%%%%%%%%%%%%%%%%%%%%%%%%%%%%%%%%%%%%%%%%%%%%%%%%%%%%%%%%%%%%%%%%%%%%%%%%%%%%%%%%%%%%%%%%%%%%%%    
        %%%%%%%%%%%%%%%%%%%%%%%%%%%%%%%%%%%%%%%%%%%%%%%%%%%%%%%%%%%%%%%%%%%%%%%%%%%%%%%%%%%%%%%%%%%%%%%%%%%%    
        %%%%%%%%%%%%%%%%%%%%%%%%%%%%%%%%%%%%%%%%%%%%%%%%%%%%%%%%%%%%%%%%%%%%%%%%%%%%%%%%%%%%%%%%%%%%%%%%%%%%    
        %%%%%%%%%%%%%%%%%%%%%%%%%%%%%%%%%%%%%%%%%%%%%%%%%%%%%%%%%%%%%%%%%%%%%%%%%%%%%%%%%%%%%%%%%%%%%%%%%%%%    
    \item[]
        \textbf{Question 9.2.15} \\
        If $X_1, X_2, \ldots , X_n$ and $Y_1, Y_2, \ldots , Y_m$ are independent random samples
        from normal distribution with the same $\sigma^2$,
        prove that their pooled sample variance, $S_p^2$, is an unbiased estimator for $\sigma^2$. \\
        \textbf{Solution} \\
        From \textbf{Theorem 9.2.1}, we have:
        \begin{align*}
            S_p^2 & = \frac{(n-1)S_X^2 + (m-1)S_Y^2}{n+m-2} \\
        \end{align*}
        Since $X_1, X_2, \ldots , X_n$ and $Y_1, Y_2, \ldots , Y_m$ are independent random samples
        from normal distribution with the same $\sigma^2$, we have:
        \begin{align*}
            E(S_p^2)              & = \frac{\sum_{i=1}^{n-1}\sigma^2 + \sum_{i=1}^{m-1}\sigma^2}{n+m-2} \\
                                  & = \frac{n\sigma^2 + m\sigma^2 - 2\sigma^2}{n+m-2}                   \\
            \boldsymbol{E(S_p^2)} & = \boldsymbol{\sigma^2} \qquad \text{Q.E.D}
        \end{align*}
        %%%%%%%%%%%%%%%%%%%%%%%%%%%%%%%%%%%%%%%%%%%%%%%%%%%%%%%%%%%%%%%%%%%%%%%%%%%%%%%%%%%%%%%%%%%%%%%%%%%%    
        %%%%%%%%%%%%%%%%%%%%%%%%%%%%%%%%%%%%%%%%%%%%%%%%%%%%%%%%%%%%%%%%%%%%%%%%%%%%%%%%%%%%%%%%%%%%%%%%%%%%    
        %%%%%%%%%%%%%%%%%%%%%%%%%%%%%%%%%%%%%%%%%%%%%%%%%%%%%%%%%%%%%%%%%%%%%%%%%%%%%%%%%%%%%%%%%%%%%%%%%%%%    
        %%%%%%%%%%%%%%%%%%%%%%%%%%%%%%%%%%%%%%%%%%%%%%%%%%%%%%%%%%%%%%%%%%%%%%%%%%%%%%%%%%%%%%%%%%%%%%%%%%%%    
        %%%%%%%%%%%%%%%%%%%%%%%%%%%%%%%%%%%%%%%%%%%%%%%%%%%%%%%%%%%%%%%%%%%%%%%%%%%%%%%%%%%%%%%%%%%%%%%%%%%% 
    \item[]
        \textbf{Question 9.2.17} \\
        \begin{tabular}{c c c c}
            \hline
            $n_i$ & X    & $m_i$ & Y   \\
            \hline
            1     & 640  & 12    & 10  \\

            2     & 80   & 13    & 320 \\

            3     & 1280 & 14    & 320 \\

            4     & 160  & 15    & 320 \\

            5     & 640  & 16    & 320 \\

            6     & 640  & 17    & 80  \\

            7     & 1280 & 18    & 160 \\

            8     & 640  & 19    & 10  \\

            9     & 160  & 20    & 640 \\

            10    & 320  & 21    & 160 \\

            11    & 160  & 22    & 320 \\
            \hline
        \end{tabular} \\ \\
        $H_0: \mu_X = \mu_Y \qquad H_1: \mu_X \neq \mu_Y$ \\
        $\alpha = 0.05$ \\
        $S_X = 428, \quad S_Y = 183$ \\ \\
        \textbf{Solution} \\
        $\bar{X} = \frac{1}{n}\sum_{i=1}^{n}X_i = 545.45$ \\
        $\bar{Y} = \frac{1}{m}\sum_{i=1}^{m}Y_i = 241.82$ \\
        From \textbf{Theorem 9.2.1}, we can find $S_p$ as follows:
        \begin{align*}
            S_p^2        & = \frac{(n-1)S_X^2 + (m-1)S_Y^2}{n+m-2}                                                     \\
                         & = \frac{(11-1)428^2 + (11-1)183^2}{11+11-2}                                                 \\
                         & = 108336.5                                                                                  \\
            S_P          & \approx 329.1                                                                               \\
            \hat{\theta} & \approx \frac{S_X^2}{S_Y^2} = \frac{428^2}{183^2} = 5.47                                    \\
            13.54        & \approx \frac{(5.47 + \frac{11}{11})^2}{\frac{1}{10}5.47^2 + \frac{1}{10}(\frac{11}{11})^2} \\
            v            & = 14
        \end{align*}
        From \textbf{Theorem 9.2.2}, the test statistic is as follows:
        \begin{align*}
            t              & = \frac{\bar{X} - \bar{Y}}{S_p\sqrt{\frac{1}{n} + \frac{1}{m}}}         \\
                           & = \frac{(545.45 - 241.82) - 0}{329.1\sqrt{\frac{1}{11} + \frac{1}{11}}} \\
                           & = 2.164                                                                 \\
            t              & > t_{\alpha/2, df}                                                      \\
                           & > t_{0.05, 14}                                                          \\
            \boldsymbol{t} & > \boldsymbol{1.7613} \qquad \textbf{Reject } \boldsymbol{H_0}          \\
        \end{align*}
        %%%%%%%%%%%%%%%%%%%%%%%%%%%%%%%%%%%%%%%%%%%%%%%%%%%%%%%%%%%%%%%%%%%%%%%%%%%%%%%%%%%%%%%%%%%%%%%%%%%%    
        %%%%%%%%%%%%%%%%%%%%%%%%%%%%%%%%%%%%%%%%%%%%%%%%%%%%%%%%%%%%%%%%%%%%%%%%%%%%%%%%%%%%%%%%%%%%%%%%%%%%    
        %%%%%%%%%%%%%%%%%%%%%%%%%%%%%%%%%%%%%%%%%%%%%%%%%%%%%%%%%%%%%%%%%%%%%%%%%%%%%%%%%%%%%%%%%%%%%%%%%%%%    
        %%%%%%%%%%%%%%%%%%%%%%%%%%%%%%%%%%%%%%%%%%%%%%%%%%%%%%%%%%%%%%%%%%%%%%%%%%%%%%%%%%%%%%%%%%%%%%%%%%%%    
        %%%%%%%%%%%%%%%%%%%%%%%%%%%%%%%%%%%%%%%%%%%%%%%%%%%%%%%%%%%%%%%%%%%%%%%%%%%%%%%%%%%%%%%%%%%%%%%%%%%%    
    \item[]
        \textbf{Question 9.2.18} \\
        For the approximate two-sample $t$ test described in Question 9.2.17, it will be true that
        \begin{align*}
            v < n + m -2
        \end{align*}
        Why is that a disadvantage for the approximate test?
        That is, why is it better to use the Theorem 9.2.1 version of the $t$ test,
        in fact, $\sigma^2_X = \sigma^2_Y$? \\
        \textbf{Solution} \\
        Because the sample of collected subjects belongs to a normal distribution population.
        %%%%%%%%%%%%%%%%%%%%%%%%%%%%%%%%%%%%%%%%%%%%%%%%%%%%%%%%%%%%%%%%%%%%%%%%%%%%%%%%%%%%%%%%%%%%%%%%%%%%    
        %%%%%%%%%%%%%%%%%%%%%%%%%%%%%%%%%%%%%%%%%%%%%%%%%%%%%%%%%%%%%%%%%%%%%%%%%%%%%%%%%%%%%%%%%%%%%%%%%%%%    
        %%%%%%%%%%%%%%%%%%%%%%%%%%%%%%%%%%%%%%%%%%%%%%%%%%%%%%%%%%%%%%%%%%%%%%%%%%%%%%%%%%%%%%%%%%%%%%%%%%%%    
        %%%%%%%%%%%%%%%%%%%%%%%%%%%%%%%%%%%%%%%%%%%%%%%%%%%%%%%%%%%%%%%%%%%%%%%%%%%%%%%%%%%%%%%%%%%%%%%%%%%%    
        %%%%%%%%%%%%%%%%%%%%%%%%%%%%%%%%%%%%%%%%%%%%%%%%%%%%%%%%%%%%%%%%%%%%%%%%%%%%%%%%%%%%%%%%%%%%%%%%%%%% 
    \item[]
        \textbf{Question 9.3.3} \\

        \textbf{Solution}
        \begin{itemize}
            \item[(a)]
                We have,
                \begin{align*}
                    n                  & = 20, \qquad \bar{x} = 3.55                         \\
                    m                  & = 20, \qquad \bar{y} = 2.10                         \\
                    \sum_{1}^{20}x_i^2 & = 319.0, \qquad S_X^2 = 1.877                       \\
                    \sum_{1}^{20}y_i^2 & = 134.0, \qquad S_Y^2 = 1.553                       \\
                    H_0: \sigma_X^2    & = \sigma_Y^2 \qquad H_1: \sigma_X^2 \neq \sigma_Y^2 \\
                    \alpha             & = 0.05
                \end{align*}
                Applying \textbf{Theorem 9.3.1}, we have:
                \begin{align*}
                    \frac{S_Y^2}{S_X^2}              & = \frac{1.553^2}{1.877^2} = 0.684                                              \\
                                                     & > F_{\alpha/2, m-1, n-1} > F_{0.025, 19, 19} = 0.406                           \\
                    \boldsymbol{\frac{S_Y^2}{S_X^2}} & > \boldsymbol{F_{\alpha/2, m-1, n-1}} \qquad \textbf{Accept } \boldsymbol{H_0}
                \end{align*}
            \item[(b)]
                \begin{align*}
                    S_p^2          & = \frac{(n-1)S_X^2 + (m-1)S_Y^2}{n+m-2}                                  \\
                                   & = \frac{(20-1)1.877^2 + (20-1)1.553^2}{20+20-2}                          \\
                                   & = 2.94                                                                   \\
                    S_p            & \approx 1.714                                                            \\
                    t              & = \frac{\bar{x} - \bar{y}}{S_p\sqrt{\frac{1}{n} + \frac{1}{m}}}          \\
                                   & = \frac{3.55 - 2.10}{1.714\sqrt{\frac{1}{20} + \frac{1}{20}}}            \\
                                   & = 2.675                                                                  \\
                    t              & > t_{\alpha/2, 38} = 2.0244                                              \\
                    \boldsymbol{t} & > \boldsymbol{t_{\alpha/2, df}} \qquad \textbf{Reject } \boldsymbol{H_0}
                \end{align*}
        \end{itemize}
        %%%%%%%%%%%%%%%%%%%%%%%%%%%%%%%%%%%%%%%%%%%%%%%%%%%%%%%%%%%%%%%%%%%%%%%%%%%%%%%%%%%%%%%%%%%%%%%%%%%%    
        %%%%%%%%%%%%%%%%%%%%%%%%%%%%%%%%%%%%%%%%%%%%%%%%%%%%%%%%%%%%%%%%%%%%%%%%%%%%%%%%%%%%%%%%%%%%%%%%%%%%    
        %%%%%%%%%%%%%%%%%%%%%%%%%%%%%%%%%%%%%%%%%%%%%%%%%%%%%%%%%%%%%%%%%%%%%%%%%%%%%%%%%%%%%%%%%%%%%%%%%%%%    
        %%%%%%%%%%%%%%%%%%%%%%%%%%%%%%%%%%%%%%%%%%%%%%%%%%%%%%%%%%%%%%%%%%%%%%%%%%%%%%%%%%%%%%%%%%%%%%%%%%%%    
        %%%%%%%%%%%%%%%%%%%%%%%%%%%%%%%%%%%%%%%%%%%%%%%%%%%%%%%%%%%%%%%%%%%%%%%%%%%%%%%%%%%%%%%%%%%%%%%%%%%% 
    \item[]
        \textbf{Question 9.5.7} \\

        \textbf{Solution}
        We have,
        \begin{align*}
            n                 & = 8, \qquad \bar{x} = 63.25          \\
            m                 & = 6, \qquad \bar{y} = 44.67          \\
            \sum_{1}^{8}x_i^2 & = 32966.0, \qquad S_X^2 = 137.367    \\
            \sum_{1}^{6}y_i^2 & = 13672.0, \qquad S_Y^2 = 340.277    \\
            H_0: \mu_X        & = \mu_Y \qquad H_1: \mu_X \neq \mu_Y \\
            \alpha            & = 0.05
        \end{align*}
        Applying \textbf{Theorem 9.2.2}, we have:
        \begin{align*}
            S_p            & = \sqrt{\frac{\sum_{1}^{n}(x_i - \bar{x})^2 + \sum_{1}^{m}(y_i - \bar{y})^2}{n+m-2}} \\
                           & = \sqrt{\frac{32966.0 + 13672.0}{8+6-2}}                                             \\
                           & \approx 14.897                                                                       \\
            t              & = \frac{\bar{x} - \bar{y}}{S_p\sqrt{\frac{1}{n} + \frac{1}{m}}}                      \\
                           & = \frac{63.25 - 44.67}{14.897\sqrt{\frac{1}{8} + \frac{1}{6}}}                       \\
                           & = 2.3099                                                                             \\
            t              & > t_{\alpha/2, df} = t_{0.025, 12} = 2.1788                                          \\
            \boldsymbol{t} & \boldsymbol{> t_{\alpha/2, df}} \qquad \textbf{Reject } \boldsymbol{H_0}
        \end{align*}
        Applying \textbf{Theorem 9.5.2}, we have:
        \begin{align*}
             & = \big( \frac{S_X^2}{S_Y^2}F_{\alpha/2, n-1, m-1}, \frac{S_X^2}{S_Y^2}F_{1-\alpha/2, n-1, m-1} \big) \\
             & = \big( \frac{137.367}{340.277}F_{0.025, 7, 5}, \frac{137.367}{340.277}F_{0.975, 7, 5} \big)         \\
             & = \big( \frac{137.367}{340.277}0.146, \frac{137.367}{340.277}0.529 \big)                             \\
             & = (0.0589, 0.2136)
        \end{align*}
        It is correct to use 9.2.2 because the ratio $\frac{\sigma_X^2}{\sigma_Y^2}$ is within the range of the confidence interval.
        %%%%%%%%%%%%%%%%%%%%%%%%%%%%%%%%%%%%%%%%%%%%%%%%%%%%%%%%%%%%%%%%%%%%%%%%%%%%%%%%%%%%%%%%%%%%%%%%%%%%    
        %%%%%%%%%%%%%%%%%%%%%%%%%%%%%%%%%%%%%%%%%%%%%%%%%%%%%%%%%%%%%%%%%%%%%%%%%%%%%%%%%%%%%%%%%%%%%%%%%%%%    
        %%%%%%%%%%%%%%%%%%%%%%%%%%%%%%%%%%%%%%%%%%%%%%%%%%%%%%%%%%%%%%%%%%%%%%%%%%%%%%%%%%%%%%%%%%%%%%%%%%%%    
        %%%%%%%%%%%%%%%%%%%%%%%%%%%%%%%%%%%%%%%%%%%%%%%%%%%%%%%%%%%%%%%%%%%%%%%%%%%%%%%%%%%%%%%%%%%%%%%%%%%%    
        %%%%%%%%%%%%%%%%%%%%%%%%%%%%%%%%%%%%%%%%%%%%%%%%%%%%%%%%%%%%%%%%%%%%%%%%%%%%%%%%%%%%%%%%%%%%%%%%%%%% 
\end{enumerate}
%%%%%%%%%%%%%%%%%%%%%%%%%%%%%%%%%%%%%%%%%%%%%%%%%%%%%%%%%%%%%%%%%%%%%%%%%%%%%%%%%%%%%%%%%%%%%%%%%%%%    
\end{document}